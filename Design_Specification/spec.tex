\documentclass[12pt]{article}

\usepackage{graphicx}
\usepackage{paralist}
\usepackage{amsfonts}
\usepackage{amsmath}
\usepackage{hhline}
\usepackage{booktabs}
\usepackage{multirow}
\usepackage{multicol}
\usepackage{url}
\usepackage{hyperref}
\oddsidemargin 0mm
\evensidemargin 0mm
\textwidth 160mm
\textheight 200mm

\renewcommand\baselinestretch{1.0}

\pagestyle {plain}
\pagenumbering{arabic}

\newcounter{stepnum}

%% Comments

\usepackage{color}

\newif\ifcomments\commentstrue

\ifcomments
\newcommand{\authornote}[3]{\textcolor{#1}{[#3 ---#2]}}
\newcommand{\todo}[1]{\textcolor{red}{[TODO: #1]}}
\else
\newcommand{\authornote}[3]{}
\newcommand{\todo}[1]{}
\fi

\newcommand{\wss}[1]{\authornote{blue}{SS}{#1}}

\title{Assignment 4, 2048 Game Design}
\author{Tingyu Shi - shit19}
\date{2021 - 04 - 12}

\begin{document}

\maketitle

This MIS contains the specific information of modules for the 
implementation of game 2048. To make this game more 
general, the user can decide the size of the board at the 
beginning of the game. The following are the rules of the game:\\
The goal of this game is to combine numbered tiles to gain higher
numbered tiles. Two numbered tiles can only be combined or merged
when two numbers are the same. The result of combining two same 
numbered tiles is the sum of these two numbers. During the game,
there are only four movements, which are left, right, up and down.
When the player chooses a direction to move, all the numbered tiles
will move at the same time. If the game board changes after a move,
a random number will pop up in the game board.\\

An on-line version of 2048 can be found at:
\begin{center}
\href {https://play2048.co/} {https://play2048.co/}
\end{center}

\section{Likely changes}
\begin{enumerate}
\item The size of the board may change
\item The printed message on screen may change
\item The data structure to store the game board may change
\item The random numbers showed in the game board after a move may change
\end{enumerate}
\section{Overview of the design}
To design this game, MVC(Model View Control) design pattern is used.
The following are specific information of each part:
\begin{itemize}
\item Model\\
The model of 2048 game is obviously the game board. This 
corresponds to the \verb|BoardT| class in the implementation. This 
class is responsible for storing  data and states of a game board.
The data is the numbered tiles of in the game board. The states 
include the score of the current game and the size of the game board.
The data structure to store the game board is a 2D array. Also, in this class, there are methods to perform different move
operations, determine if the game has failed.etc. For detailed 
information, please check MIS of \verb|BoardT|.
\item View\\
View unit corresponds to \verb|DisplayGame| code file. The role of 
this module is  to display the game board and some other messages to 
players. Other messages include game instructions, welcoming message,
ending messages.etc.
\item Control\\
The Control unit corresponds to \verb|GameControl| of the code file.
This is a link between model and view. Also, it handles user's inputs
to modify the game board and select view correspondingly. 
\item Another java class is \verb|MoveDirection|, this class is
implemented as enum class, it contains all the movement options of
game 2048.
\end{itemize}

\newpage

\section{MIS of each Module}

%%%%%%%%%MoveDirection Module%%%%%%%%%%%%%%%%
\section* {MoveDirection Module}
\subsection*{Module}
MoveDirection\\
\subsection* {Uses}
None
\subsection* {Syntax}
\subsubsection* {Exported Constants}
None
\subsubsection* {Exported Types}
MoveDirection = \{\\
    LEFT, \textit{\#Move all the numbered tiles left}\\
    RIGHT, \textit{\#Move all the numbered tiles right}\\
    UP, \textit{\#Move all the numbered tiles up}\\
    DOWN, \textit{\#Move all the numbered tiles down}\\
\}
\subsubsection* {Exported Access Programs}
None
\subsection* {Semantics}
\subsubsection* {State Variables}
None
\subsubsection* {State Invariant}
None
\subsubsection* {Assumptions}
None
\newpage




%%%%%%%%%%%%%%%%%%BoardT module
\section* {Game Board Module}

\subsection*{Template Module}

BoardT\\

\subsection* {Uses}

MoveDirection

\subsection* {Syntax}

\subsubsection* {Exported Constants}

None

\subsubsection* {Exported Types}

BoardT = ?

\subsubsection* {Exported Access Programs}

\begin{tabular}{| l | l | l | p{5cm} |}
  \hline
  \textbf{Routine name} & \textbf{In} & \textbf{Out} & \textbf{Exceptions}\\
  \hline
  new BoardT & $\mathbb{N}$ & BoardT & IllegalArgumentException \\
  \hline
  move & MoveDirection & $\mathbb{B}$ & ~\\
  \hline
  put\_one\_random\_number & $\mathbb{N}$ &  & ~\\
  \hline
  is\_end & & $\mathbb{B}$ & ~\\
  \hline
  set\_board & $\mathbb{N}$, $\mathbb{N}$, $\mathbb{N}$ & & IllegalArgumentException\\
  \hline
  has\_empty\_cell &  & $\mathbb{B}$ & ~\\
  \hline
  get\_score &  & $\mathbb{N}$ & ~\\
  \hline
  get\_board &  & seq of (seq of $\mathbb{N}$) &~\\
  \hline
  get\_size &&$\mathbb{N}$& ~\\
  \hline
  equals &seq of (seq of $\mathbb{N}$)&$\mathbb{B}$&IllegalArgumentException\\
  \hline
    
\end{tabular}

\subsection* {Semantics}

\subsubsection* {State Variables}

$\mathit{size}: \mathbb{N}$\\
$\mathit{score}: \mathbb{N}$\\
$\mathit{board}$: seq[$\mathit{size}$] of (seq[$\mathit{size}$] of $\mathbb{N}$) \\ \textit{\# This is a 2D array to store all the data of the game board, the game board is a square board}

\subsubsection* {State Invariant}

None

\subsubsection* {Access Routine Semantics}

\noindent new BoardT($\mathit{board\_size}$):
\begin{itemize}
\item transition:\\
$\mathit{size} := \mathit{board\_size}$\\
$\mathit{score} := 0$\\
$\mathit{board}$ $:=$ 
      $\langle \begin{array}{c}
      \langle \mbox{0}_{00}, \mbox{0}_{01} ... ... ,\mbox{0}_{0(\mathit{size} - 2)}, \mbox{0}_{0(\mathit{size} - 1)} \rangle\\
      
      \langle \mbox{0}_{10}, \mbox{0}_{11} ... ... ,\mbox{0}_{1(\mathit{size} - 2)}, \mbox{0}_{1(\mathit{size} - 1)} \rangle\\
      
      \langle \mbox{0}_{20}, \mbox{0}_{21} ... ... ,\mbox{0}_{2(\mathit{size} - 2)}, \mbox{0}_{2(\mathit{size} - 1)} \rangle\\
      
      
      ...\\
      ...\\
      
      \langle \mbox{0}_{(\mathit{size} - 3)0}, \mbox{0}_{(\mathit{size} - 3)1} ... ... ,\mbox{0}_{(\mathit{size} - 3)(\mathit{size} - 2)}, \mbox{0}_{(\mathit{size} - 3)(\mathit{size} - 1)} \rangle\\
      
      
      \langle \mbox{0}_{(\mathit{size} - 2)0}, \mbox{0}_{(\mathit{size} - 2)1} ... ... ,\mbox{0}_{(\mathit{size} - 2)(\mathit{size} - 2)}, \mbox{0}_{(\mathit{size} - 2)(\mathit{size} - 1)} \rangle\\
      
      
      \langle \mbox{0}_{(\mathit{size} - 1)0}, \mbox{0}_{(\mathit{size} - 1)1} ... ... ,\mbox{0}_{(\mathit{size} - 1)(\mathit{size} - 2)}, \mbox{0}_{(\mathit{size} - 1)(\mathit{size} - 1)} \rangle\\
      
      
      \end{array} \rangle$ \\ 

\textit{\#$\mathit{board}$ is a 2D array which has $\mathit{size}$ rows
and for each row, it has $\mathit{size}$ 0s. The subscript in the above graph is the index number of each 0.}

\item output: $out := \mbox{self}$
\item exception: exc := $((\mathit{board\_size} < 0 \lor \mathit{board\_size} = 0) \Rightarrow \text{IllegalArgumentException})$
\end{itemize}

%%%%%%%%%not finished
\noindent move($\mathit{dir}$):
\begin{itemize}
\item transition: transition will be made during the process of  calculating the output
\item output: $(dir = MoveDirection.LEFT \Rightarrow (move\_left(\mathit{board}))  | \\  dir = MoveDirection.RIGHT \Rightarrow (move\_right(\mathit{board})) |\\ dir = MoveDirection.UP \Rightarrow (move\_up(\mathit{board}))| \\dir = MoveDirection.DOWN \Rightarrow (move\_down(\mathit{board})))$\\
\textit{\#At the same time of calculating the output, the state of 
of the game board is changed}
\item exception: None\\
\end{itemize}

\newpage
\noindent put\_one\_random\_number($\mathit{value}$):
\begin{itemize}
\item transition: $\mathit{board}[i][j] := value$\\
\textit{\#Assum we have the function random() to generate a number between 0 and 1}.
Where
$$\langle i, j \rangle := \mathit{emptyPos}[\lfloor random()*|emptyPos| - 1 \rfloor]$$
$$\mathit{emptyPos} := \{\forall i, j : \mathbb{N} | i \in [0..\mathit{size}-1] \land j \in [0..\mathit{size}-1] \land \mathit{board}[i][j] = 0 : \langle i, j \rangle\}$$\\
\textit{\#emptyPos here is a set of pairs (i, j) such that board[i][j] = 0. Then we just random choose a pair and assign value to board[i][j].}
\item output: none
\item exception: none
\end{itemize}

%%%%%%%%not finished
\noindent is\_end():
\begin{itemize}
\item transition: none
\item output: $\mathit{out} := \lnot (\mathit{move}(MoveDirection.LEFT) \lor \mathit{move}(MoveDirection.RIGHT) \lor \mathit{move}(MoveDirection.UP) \lor \mathit{move}(MoveDirection.DOWN)) $\\
\textit{\#If the game board can move left, right, up, or down, then game is not end.}
\item exception: none
\end{itemize}


\noindent set\_board($\mathit{x}$, $\mathit{y}$, $\mathit{value}$):
\textit{\#This method is just used for testing.}
\begin{itemize}
\item transition:
$\mathit{board}[x][y] := value$
\item output: none
\item exception: exc := (($\mathit{x} < 0$ $\lor$ $\mathit{y} < 0$ $\lor$ $\mathit{x} \geq \mathit{size}$ $\lor$ $\mathit{y} \geq \mathit{size}$) $\Rightarrow$ \text{IllegalArgumentException})
\end{itemize}

\noindent has\_empty\_cell():
\begin{itemize}
\item transition: none
\item output: 
 $$out := \exists(i, j: \mathbb{N} | i \in [0..\mathit{size}-1] \land j \in [0..\mathit{size}-1] : \mathit{board}[i][j] = 0)$$\textit{\#Used to tell if board has 0}
\item exception: none
\end{itemize}

\noindent get\_score():
\begin{itemize}
\item transition: none
\item output: $\mathit{out} := \mathit{score}$
\item exception: none
\end{itemize}

\noindent get\_board():
\begin{itemize}
\item transition: none
\item output: $\mathit{out} := \mathit{board}$
\item exception: exc := none
\end{itemize}

\noindent get\_size():
\begin{itemize}
\item transition: none
\item output: $\mathit{out} := \mathit{size}$
\item exception: none
\end{itemize}

\noindent equals($\mathit{arr}$):
\textit{\#This method is just used for testing.}
\begin{itemize}
\item transition: none
\item output:$\mathit{out} := \lnot \exists (i, j : \mathbb{N} | i \in [0..\mathit{size}-1] \land j \in [0..\mathit{size}-1] : \mathit{board}[i][j] \neq \mathit{arr}[i][j])$
\item exception: $exc := ( \lnot (|arr| = \mathit{size}) \lor \lnot(|arr[0]| = \mathit{size})) \Rightarrow IllegalArgumentException$
\end{itemize}

\newpage
\subsection*{Local Functions}
\noindent \textit{\#For the following local functions, operational descriptions will be used since it is hard to express in mathematical expression.}\\
\noindent $\mbox{move\_left} : \text{seq} [\mathit{size}] \text{ of } (\text{seq} [\mathit{size}] \text{ of } \mathbb{N}) \rightarrow \mathbb{B}$\\ The following are steps:
\begin{enumerate}
\item For each row, merge and sum up two adjacent positive numbers if they are the
same. A number will always try to merge the number at the left side 
of it if the above condition is met. Also, if a number is produced 
by the above merging, it will not merge with other numbers again.
\item After merging, for each row, moving all the positive numbers
toward left. The order between these positive numbers must not change.
\item After the above two steps, compare the current 2D array with
the input 2D array, if they are the same, return true, otherwise 
return false.
\end{enumerate}

\noindent $\mbox{move\_right} : \text{seq} [\mathit{size}] \text{ of } (\text{seq} [\mathit{size}] \text{ of } \mathbb{N}) \rightarrow \mathbb{B}$\\ It is the same as $\mbox{move\_left}$ except in step2, 
moving all the positive numbers toward right.\\

\noindent $\mbox{move\_up} : \text{seq} [\mathit{size}] \text{ of } (\text{seq} [\mathit{size}] \text{ of } \mathbb{N}) \rightarrow \mathbb{B}$\\ The following are steps:
\begin{enumerate}
\item For each column, merge and sum up two adjacent positive numbers if they are the
same. A number will always try to merge the number at the top of it if the above condition is met. Also, if a number is produced 
by the above merging, it will not merge with other numbers again.
\item After merging, for each column, moving all the positive numbers
up. The order between these positive numbers must not change.
\item After the above two steps, compare the current 2D array with
the input 2D array, if they are the same, return true, otherwise 
return false.
\end{enumerate}


\noindent $\mbox{move\_down} : \text{seq} [\mathit{size}] \text{ of } (\text{seq} [\mathit{size}] \text{ of } \mathbb{N}) \rightarrow \mathbb{B}$\\ It is the same as $\mbox{move\_up}$ except in step 2, 
moving all the positive numbers down.\\
\newpage





%%%%%%%%%%%%%%%%%%%%%%%%%%%%%%%%%%%%%%%%
%DisplayGame module
%%%%%%%%%%%%%%%%%%%%%%%%%%%%%%%%%%%%%%%%
\section* {DisplayGame Module}

\subsection* {DisplayGame Module}

\subsection* {Uses}

None

\subsection* {Syntax}

\subsubsection* {Exported Types}

None

\subsubsection* {Exported Constants}

None

\subsubsection* {Exported Access Programs}

\begin{tabular}{| l | l | l | p{6cm} |}
\hline
\textbf{Routine name} & \textbf{In} & \textbf{Out} & \textbf{Exceptions}\\
\hline
print\_board & BoardT &  &  \\
\hline
print\_game\_start & ~ & ~ & \\
\hline
print\_game\_instruction & ~ & ~ & \\
\hline
print\_game\_ending& ~ & ~ & \\
\hline
print\_invlid\_command & ~ & ~ & \\
\hline
print\_game\_fail & ~ & ~ & \\
\hline
print\_wrong\_size\_input & ~ & ~ & \\
\hline
\end{tabular}

\subsection* {Semantics}

\subsection*{Environment Variables}

window: A portion of computer screen to display the game and messages

\subsubsection* {State Variables}

None

\subsubsection* {State Invariant}

None

\subsubsection* {Access Routine Semantics}

\noindent print\_board($\mathit{board}$):
\begin{itemize}
\item transition: window := Draws the game board onto the screen.
Each elements in the board is accessed by using \verb|get_board| 
method defined in \verb|BoardT| class. Assume the size of the board 
is s, then the first row is from $\mathit{board}$[0][0] to $\mathit{board}$[0][s - 1], the second row will be from $\mathit{board}$[1][0] to $\mathit{board}$[1][s - 1]. Follow this
pattern, the last row will be from $\mathit{board}$[s - 1][0] to $\mathit{board}$[s - 1][s - 1].
\end{itemize}

\noindent print\_game\_start():
\begin{itemize}
\item transition: window $:=$ Displays a welcome message when the user starts the game.
\item output: none
\item exception: none
\end{itemize}

\noindent print\_game\_instructions():
\begin{itemize}
\item transition: window := After game starts, this will print game
instructions onto the screen. Instruction include how to control different
movements and how to exit the game. Also, the window shows a message
to ask the player which movement he/she wants to play and whether 
he/she wants to exit the game.
\item output: none
\item exception: none
\end{itemize}

\noindent print\_game\_ending():
\begin{itemize}
\item transition: window $:=$ If the player chooses to exit the game,
just print an ending message.
\item output: none
\item exception: none
\end{itemize}

\noindent print\_invalid\_command():
\begin{itemize}
\item transition: window $:=$ Remind the user that the input
 command is 
invalid and shows a message to tell the user to enter a new command.
\item output: none
\item exception: none
\end{itemize}

\noindent print\_game\_fail():
\begin{itemize}
\item transition: window := Print a message when the game fails, 
which means no more moves can be made.
\item output: none
\item exception: none
\end{itemize}

\noindent print\_wrong\_size\_input():
\begin{itemize}
\item transition: window := Print a message if the user enters an
invalid board size. For example, 0 or negative numbers. Also, window
shows a message to tell the user to enter a new number again.
\item output: none
\item exception: none
\end{itemize}

\newpage


%%%%%%%%%%%%%%%%%%%%%%%%%%%%%%%%%%%%%%%%%%%%%%%%%%%%%%%%%%%%
%   Game control module
%%%%%%%%%%%%%%%%%%%%%%%%%%%%%%%%%%%%%%%%%%%%%%%%%%%%%%%%%%%%
\section* {GameControl Module}
\textit{\#This part of specification
is operational}

\subsection* {GameControl Module}

\subsection* {Uses}

BoardT, DisplayGame

\subsection* {Syntax}

\subsubsection* {Exported Types}

None

\subsubsection* {Exported Constants}

None

\subsubsection* {Exported Access Programs}

\begin{tabular}{| l | l | l | p{4.7cm} |}
\hline
\textbf{Routine name} & \textbf{In} & \textbf{Out} & \textbf{Exceptions}\\
\hline
initialize\_game & ~ & BoardT & ~ \\
\hline
run\_game & BoardT & ~ & ~\\
\hline
main & ~ & ~ & ~ \\
\hline
\end{tabular}

\subsection* {Semantics}

\subsection*{Environment Variables}

keyboard: Scanner(System.in) \qquad \textit{// reading inputs from players via keyboards}

\subsubsection* {State Variables}

None
\subsubsection* {State Invariant}

None

\subsubsection* {Access Routine Semantics}

initialize\_game():
\begin{itemize}
  \item transition: operational method for initializing the game.
  \begin{enumerate}
  \item Print the welcome message.
  \item Asking the player to enter the board size he/she wants to use and then receive the board size from the keyboard.
  \item Create an object of BoardT with the size received 
  from step 2.
  \item Put 2 numbers(2 and 2 or 2 and 4) into two randomly chosen 
  positions on the board.
  \item Display the board onto the screen.
  \item Return the board created from step 3.
  \end{enumerate}
  \item output: out := The BoardT object created from step 3 above.
  \item exception: None
\end{itemize}

\noindent run\_game(board):
\begin{itemize}
  \item transition: operational method for running the game.\\ After
  printing the initialized game board onto the screen, the game will
  display the instruction and ask the user to make a move or exit the
  game. If the player chooses to make a move, then the game will
  display the game board after that move. If the player chooses to
  exit the game, the game will end by displaying the ending message.
  The game will continue as above until no more moves can be made.
  \item output: None
  \item exception: None
\end{itemize}

\noindent main(): \textit{\#This is the main function. The driver of the whole game}
\begin{itemize}
  \item transition: operational method for running the game.
  \begin{enumerate}
  \item BoardT model = initialize\_game(); \textit{ //Initialize the game}
  \item run\_game(model); \textit{ // run game}
  \end{enumerate}
  \item output: None
  \item exception: None
\end{itemize}
\newpage

\section{Critique of Design}
\begin{itemize}
\item Consistency\\
The consistency of this design is met. This includes naming
conventions and ordering of parameters and exception handling. 
\item Essentiality\\
There are two methods from \verb|BoardT| class are not essential, 
which are \verb|set_board| and \verb|equals|. These two methods are
not essential for implementing the game, however, they are  useful for testing
the model, therefore, they can be deleted after testing.
\item Generality\\
For a standard 2048 game, the game board is 4 $\times$ 4. However, 
this design has improved the generality so that the player can set
up a n $\times$ n board. Also, because of \verb|put_one_random_number| method from \verb|BoardT| class, we can put different random values after each move, not only 2 or 4. However,
the generality can still be improved, the next improvement can be
different shapes of the board, not only square board.
\item Minimality\\
The Minimality should be improved for \verb|move| method of class \verb|BoardT|. The reason is that \verb|move| provides two 
independent services. First, it changes the state of the game board 
according to the MoveDirection given. At the same time, it returns
a boolean to tell if the board has changed before and after the move
operation. The correct way of designing should be make a copy of 
the current board and define another method to compare the board after
the move and copy of the board before the move.
\item Cohesion\\
The cohesion of this design is achieved. The reason is that the design
follows the structure of MVC. Each module except \verb|MoveDirection| just implements one of the units of MVC. As a consequence, the components within one module are highly related.
\item Information Hiding\\
All the state variables are private variables. However, since we need
\verb|set_board| method for testing, the programmer can still change
the state of game board from outside. Therefore, we need to come up with a better way for testing when there are random numbers. Also,
it is a good practice to just remove this method after testing.
\end{itemize}
\newpage
The following are some other points about this design:
\begin{enumerate}
\item  The view module are designed to be an abstract object just 
like a service module, the reason is that different game can use the 
same view module, therefore, there is not need to create an ADT.
\item About the testing, all the exceptions are tested and different
move operations are tested. However, for each move, it is better to
include more testing cases.
\end{enumerate}
\end{document}
